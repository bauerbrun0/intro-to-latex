\documentclass[a4paper,12pt]{article}
\usepackage[T1]{fontenc}
\PassOptionsToPackage{defaults=hu-min}{magyar.ldf}
\usepackage[magyar]{babel}
\begin{document}
A halmazelmélet fiatal tudományág, létrejötte a XIX.~század második
felére tehető, ami nem véletlen, hiszen a halmazok vizsgálatához nagyfokú
absztrakció szükséges.

Ekkorra értek el a matematikai kutatások olyan szin-
tet, hogy az ilyen absztrakció szükségessé és lehetővé vált. Az előzmények
közül a következő három a legfontosabb.

\begin{enumerate}
\item A matematikusok figyelme a halmazok elemeiről a halmazokra irányult.
Olyan problémák vezettek ide, melyeket bizonyos halmazokra anélkül
sikerült megoldani, hogy azokat az egyes halmazelemekre vonatkoztat-
ták volna (pl.~biztosítási matematika, kinetikus gázelmélet).

\item A kritikai szellem fejlődése, ami azt jelentette, hogy részletesen ele-
mezték a korábban magától értetődőnek és ezért általános érvényűnek
tekintett megállapításokat. (Ennek nagy szerepe volt a matematikai
logika fejlődésében is.)

\item A legdöntőbb momentum az volt, amikor a végtelen sorok vizsgálata
közben felismerték, hogy a véges halmazok tulajdonságaival nem ren-
delkeznek törvényszerűen a \emph{végtelen halmazok} is.
\end{enumerate}
\end{document}
